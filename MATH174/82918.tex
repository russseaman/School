\documentclass{article}
\title{MATH 174 Class Notes}
\author{Russ Seaman}
\begin{document}
	\maketitle
	\tableofcontents
	\section{Chapter 1}
	\subsection[1.1]{1.1}
	\subsection[1.2]{1.2}

	\[A\ *\ B = {(a,b)\ |\ a \in A\ and\ b \in B}\]
	
	note:
	\[A\ *\ B\ \!= B\ *\ A \]
	\subsubsection{Homework Examples}
	5.)	\[A = \{0,1,2\}	\]
	
		\[B = \{x \in \mathbf{R}\ |-1 <= x< 3 \}\]
	\\
	\\
	\\
	
	\section{Chapter 2}
	\subsection[2.1 - Logical Operators]{2.1 - Logical Operators}
	A statement (or proposition) is a declarative sentence which is either true or false(not both).
	\\
	\bigskip
	A compound statement is a statement formed from existing statements using logical operators (see below).
	\\
	\bigskip
	The truth value of a compound statement depends on the truth values of the individual component statements that make it up.
	

\end{document}