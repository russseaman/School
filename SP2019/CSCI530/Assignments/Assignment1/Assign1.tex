\documentclass[]{article}
\usepackage{qtree}
\usepackage{amsmath}
\usepackage{mathtools}
\usepackage{graphicx}
\usepackage{enumerate}
\usepackage[most]{tcolorbox}
\usepackage[T1]{fontenc}
\usepackage[margin=10pt]{geometry}
%opening
\title{Assignment 1}
\author{Russ Seaman}

\begin{document}
\raggedright
\maketitle
Homework for chapter 3(pg.158): Questions 7(d), 8, 9 ,11\\
\textbf{Example 3.4:}
\begin{equation*} \label{e34}
\begin{split}
<assign> & \rightarrow <id> = <expr>\\
<id> & \rightarrow A | B | C\\
<expr> & \rightarrow <expr> + <term> | <term>\\
<term> & \rightarrow <term> * <factor> | <factor>\\
<factor> & \rightarrow ( <expr> )  | <id>
\end{split}
\end{equation*}

\section*{Problem 7(d):}
Using the grammar in Example 3.4, show a parse tree and a leftmost derivation for each of the following statements:\\
(d): A = B * (C * (A + B))

\begin{tcbraster}[raster columns = 2, raster equal height, raster column skip = .5cm]
	\begin{tcolorbox}[title = Parse Tree]
		\Tree[[ A ].<id> = [[[[[ B ].<id> ].<factor> ].<term> * [ ( [[[[[ C ].<id> ].<factor> ].<term> * [ ( [[[[[ A ].<id> ].<factor> ].<term> ].<expr> + [[[ B ].<id> ].<factor> ].<term> ].<expr> !{\qbalance} ) ].<factor> !{\qbalance} ].<term> !{\qbalance} ].<expr> !{\qbalance} ) ].<factor> !{\qbalance} ].<term> ].<expr> !{\qbalance} ].<assign>
	\end{tcolorbox}
	\begin{tcolorbox}[title = Left most derivation]
		\begin{align*}
		<assign> \rightarrow <id> &= <expr> \\
		\rightarrow A &= <expr>\\
		\rightarrow A &= <term>\\
		\rightarrow A &= <term>*<factor>\\
		\rightarrow A &= <factor>*<factor>\\
		\rightarrow A &= <id>*<factor>\\
		\rightarrow A &= B * <factor>\\
		\rightarrow A &= B * (<expr>)\\
		\rightarrow A &= B * (<term>)\\
		\rightarrow A &= B * (<term> * <factor>)\\
		\rightarrow A &= B * (<factor> * <factor>)\\
		\rightarrow A &= B * (<id> * <factor>)\\
		\rightarrow A &= B * (<id> * <factor>)\\
		\rightarrow A &= B * (C * <factor>)\\
		\rightarrow A &= B * (C * (<expr>))\\
		\rightarrow A &= B * (C * (<expr> + <term>))\\
		\rightarrow A &= B * (C * (<term> + <term>))\\
		\rightarrow A &= B * (C * (<factor> + <term>))\\
		\rightarrow A &= B * (C * (<id> + <term>))\\
		\rightarrow A &= B * (C * (A + <term>))\\
		\rightarrow A &= B * (C * (A + <factor>))\\
		\rightarrow A &= B * (C * (A + <id>))\\
		\rightarrow A &= B * (C * (A + B))
		\end{align*}
	\end{tcolorbox}
\end{tcbraster}

\pagebreak

\section*{Problem 8:}
Prove the following grammar is ambiguous:

\begin{align*}
	<S> &\rightarrow <A>\\
	<A> &\rightarrow <A> + <A> | <id>\\
	<id> &\rightarrow a|b|c
\end{align*}
\begin{tcbraster}[raster columns = 2, raster equal height, raster column skip = .5cm]
	\begin{tcolorbox}[title=Parse Tree 1]
		\Tree[[[[ a ].<id> ].<A> + [[[b ].<id> ].<A> + [[c ].<id> ].<A> ].<A> ].<A> ].<S>
	\end{tcolorbox}
	\begin{tcolorbox}[title=Parse Tree 2]
		\Tree[[[[[[a ].<id> ] ].<A> + [[b ].<id> ].<A> ].<A> + [[c ].<id> ].<A> ].<A> ].<S>
	\end{tcolorbox}
\end{tcbraster}
2 different trees exists for statements, therefore it is ambiguous.

\section*{Problem 9:}
Modify the grammar of Example 3.4 to add a unary minus operator that has higher precedence than either + or *.
\begin{align*}
	<assign> &\rightarrow <id> = <expr>\\
	<id> &\rightarrow A | B | C\\
	<expr> &\rightarrow <expr> - <term> | <term>\\
	<term> &\rightarrow <term>*<factor> | <factor>\\
	<factor> &\rightarrow (<expr>) | <id>
\end{align*}

	\pagebreak
\section*{Problem 11:}
Consider the following grammar:
\begin{align*}
	<S> &\rightarrow <A> a <B> b\\
	<A> &\rightarrow <A> b | B\\
	<B> &\rightarrow a <B> | a
\end{align*}
Which of the following sentences are in the language generated by this grammar?
\begin{enumerate}[a.]
	\item baab
	\begin{flalign*}
	&\rightarrow a b\\
	&\rightarrow b a b\\
	&\rightarrow b a a b\\
	\end{flalign*}
	\textbf{So, the combination will be generated from the grammar.}
	\item bbbab
	\begin{align*}
	&\rightarrow a b\\
	&\rightarrow b a b\\
	&\rightarrow b b a b\\
	&\rightarrow b b b a b\\
	&\rightarrow b b b a a b\\
	\end{align*}
\textbf{Based on the provided grammar, the combination will not be produced}

	\item bbaaaaaS
	\item bbaab

\end{enumerate}


\end{document}
